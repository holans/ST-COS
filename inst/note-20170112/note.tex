\documentclass[10pt]{article}

\usepackage{araim-common}
\usepackage{araim-tex}

\makeatletter
\def\blfootnote{\xdef\@thefnmark{}\@footnotetext}
\makeatother

\title{Note on Prediction in STCOS Model}
%\date{}
\author{Andrew M. Raim
\vspace{0.5em} \\
$^*$Center for Statistical Research and Methodology, U.S. Census Bureau
}

\begin{document}

\maketitle

% ---------------------------------------------------
\section{Introduction}
\label{sec:intro}
Recall that the process model in \citet{BradleyEtAl2016-STAT} is
%
\begin{align*}
Y_t^{(\ell)}(A) = h(A)^T \mu_B + \psi_t^{(\ell)}(A)^T \eta + \xi_t^{(\ell)}(A).
\end{align*}
%
Our first objectives are to reproduce Figures 1(g) and 1(h). These represent 2013 3-year ACS estimates and standard deviations from the model. The model-based estimate from the paper is written as
%
\begin{align*}
\hat{Y}_t^{(\ell)}(A) = \E[ Y_t^{(\ell)}(A) \mid \{ Z_t^{(\ell)}(A) \} ],
\quad A \in D_{t,A}^{(\ell)};
\quad t = T_L, \ldots, T_U, 
\quad \ell = 1,3,5.
\end{align*}
%
From this expression, the quantity $\hat{Y}_t^{(\ell)}(A)$ could be approximated using MCMC draws $\{ Y_t^{(\ell, r)}(A) : r = 1, \ldots, R \}$ as
%
\begin{align*}
\frac{1}{R} \sum_{r=1}^R Y_t^{(\ell, r)}(A).
\end{align*}
%
However, this does not help us to compute predictions for parts of space-time that did not have observations. One notable example is that the model was fit without 2013 3-year ACS data (which is being displayed in the figures of interest). Then how should we compute its predictions? Another idea is to use draws of the parameters $\{ \eta^{(r)} : r = 1, \ldots, R \}$ and $\{ \mu_B^{(r)} : r = 1, \ldots, R \}$, so that
%
\begin{align*}
\hat{Y}_t^{(\ell)}(A) = \frac{1}{R} \sum_{r=1}^R \left[ h(A)^T \mu_B^{(r)} + \psi_t^{(\ell)}(A)^T \eta^{(r)} \right].
\end{align*}
%
The complication here is that we need to compute $\psi_t^{(\ell)}(A)$ for the new space-time point, which is also not quite clear.



\bibliographystyle{plainnat}
\bibliography{references}
%\input{references.bbl}

\end{document}
