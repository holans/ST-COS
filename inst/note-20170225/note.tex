\documentclass[10pt]{article}

\usepackage{araim-common}
\usepackage{araim-tex}

\makeatletter
\def\blfootnote{\xdef\@thefnmark{}\@footnotetext}
\makeatother

\title{Computation of the Moran's I Propagator}
%\date{}
\author{Andrew M. Raim
\vspace{0.5em} \\
Center for Statistical Research and Methodology, U.S. Census Bureau
}

\begin{document}

\maketitle

% ---------------------------------------------------
\section{Introduction}
\label{sec:intro}
I noticed that in Jon's Matlab code, he didn't seem to be computing the Moran's I propagator as in the \citet{BradleyEtAl2016-STAT}. The relevant part of the code is as follows.

\begin{lstlisting}
Q = eye(3109) - 0.9*countAdj;
[PQ,LQ] = eig(Q);

% Moran's I Propagator
B = [PQ'*ones(3109,1),eye(3109)];
[M,LM] = eig(B*pinv(B'*B)*B');
M = real(M);

% Target Covariance
Kinv = make_full_model_sptcovar_9(Q,M,Sconnectorf,3109);
[P,D] = eig(Kinv);
D = real(diag(D));
D(D<0) = 0;
Dinv = D;
Dinv(D>0) = (1./D(D>0));
K = real(P)*diag(Dinv)*real(P');
Kinv = real(P)*diag(D)*real(P');
\end{lstlisting}
%
The code above leads to a Gibbs sampler that exhibits poor mixing; some alternative MCMC strategies have also failed, leading me to suspect that \code{M} may have been computed incorrectly for the ACS example. (I am also wondering about the computation of $S$, the design matrix of basis functions). The appendix in \citet{BradleyEtAl2016-STAT} specifies that \code{M} should instead be based on the eigenvectors of $I - X(X^T X)^{-1} X^T$, where $X = (H \; I)$. 


But it looks like $X(X^T X)^{-1} X^T$ is not computable. In general, for a block matrix, we have
%
\begin{align*}
\begin{bmatrix}
A & B \\
C & D
\end{bmatrix}^{-1}
=
\begin{bmatrix}
E^{-1} & -E^{-1} B D^{-1} \\
-D^{-1} C E^{-1} & D^{-1} + D^{-1} C E^{-1} B D^{-1}
\end{bmatrix}
\end{align*}
%
where $E = A - B D^{-1} C$. For our problem, we have
%
\begin{align*}
X^T X =
\begin{bmatrix}
H^T \\
I
\end{bmatrix}
\begin{bmatrix}
H & I
\end{bmatrix}
=
\begin{bmatrix}
H^T H & H^T \\
H & I
\end{bmatrix}.
\end{align*}
%
The inverse of $X^T X$ depends on inverting
%
\begin{align*}
E = H^T H - H^T I^{-1} H = 0.
\end{align*}
%
which is singular.

\bibliographystyle{plainnat}
\bibliography{references}
%\input{references.bbl}

\end{document}
