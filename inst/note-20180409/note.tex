\documentclass[12pt]{article}

\usepackage{araim-common}
\usepackage{araim-tex}

\makeatletter
\def\blfootnote{\xdef\@thefnmark{}\@footnotetext}
\makeatother

\title{An STCOS Simulation with Microdata}
%\date{}
\author{Andrew M. Raim
\vspace{0.5em} \\
Center for Statistical Research and Methodology, U.S. Census Bureau
}

\begin{document}

\maketitle

% ---------------------------------------------------
\section{Introduction}
\label{sec:intro}
We are developing simulation to validate the R implementation of the STCOS model proposed in \citet{BradleyEtAl2016-STAT}. Some initial work was presented in \citet{JSM2017-STCOS} using real ACS data; here the estimates for the source supports matched very well between the model and direct estimates, but the estimates on congressional districts did not match very well. It was not clear if the model-based estimates were problematic or not, as the direct estimates are also attempting to describe an unknown quantity. Now we are looking to generate microdata, where the truth is completely known, and compare model-based results to that. In this document, we will record details for such a proposed simulation.

\section{Simulation}
\label{sec:sim}

\begin{enumerate}
\item The fine-level support to be counties (as defined in 2015) for the state of Pennslyvania.

\item Spatial coordinates of knot-points are obtained by taking 500 points over PA using the space-filling design in the \code{fields} package. Temporal coordinates of knot-points are selected to be 2005, 2005.5, \ldots, 2014.5, 2015)

\item We define five target supports:
\begin{enumerate*}
\item[a.] Congressional districts
\item[b.] The fine-level support
\item[c.] A ``medium-sized'' grid over PA
\item[d.] A ``fine'' grid over PA, which is twice as fine as the medium-sized grid
\item[e.] A ``coarse'' grid over PA, which is half as fine as the medium-sized grid
\end{enumerate*}
%
TBD: make some pictures.

\item We use the 2016 Planning Database to get realistic means and MOEs for Median Household Income at the block-group level. We also get realistic counts of housing units at the block-group level. The three relevant variables are: \code{Med\_HHD}, \code{Med\_HHD\_MOE}, and \code{Tot\_Housing\_Units\_CEN\_2010}.

\item We generate microdata for each block group in the following way. Let $i = 1, \ldots, n_\text{BG}$ index the block group. Let the total number of housing units in the $i$th block group, $N_i$, be equal to \code{Tot\_Housing\_Units\_CEN\_2010}. For the $j$th housing unit in the $i$th block group, generate spatial coordinates from uniform distribution on the polygon for the block group. Let $\phi_i$ be equal to \code{Med\_HHD} and $\tau$ be equal to \code{Med\_HHD\_MOE / 1.645}. These quantities will not change in the simulation, but for each repetition we will redraw $y_{ij}$, the log(Median Household Income) for the $j$th household within the $i$th block group. We first assume that the Median Household Incomes observed in the Planning Database are log-normally distributed 

\sim \text{N}(\phi_i, \tau^2)$

var.normal <- log(1 + sim.popn$sd_y^2 / sim.popn$mean_y^2)
mean.normal <- log(sim.popn$mean_y) - 1/2 * var.normal
sim.popn$y <- rnorm(nrow(sim.popn), mean.normal, sqrt(var.normal))



\end{enumerate}



\appendix
\clearpage

\section{R Code Copied Over}
\label{sec:code}

\begin{footnotesize}
\begin{verbatim}
library(fields)
library(ggplot2)
library(ggforce)
library(dplyr)
library(readr)
library(sf)
library(stcos)

num.sp.knots <- 500
radius <- 1
pr.rho <- 0.75
gridn <- 16
sample.prop <- 0.005
reps <- 1

state <- 42
extract <- function(dat) {
    dat[dat$STATE == state,]
}

# ----- Load space-time domains with direct survey estimates/variances -----
# Fine-level support
county_acs_5yr2015 <- st_read("../acs_county_S1901/merged/county_acs_5yr2015.shp")
dom.fine <- extract(county_acs_5yr2015)

# ----- Set up knots for bisquare basis -----
# Knots must be compatible (same projection) with fine-level geography
u <- st_sample(dom.fine, size = 5000)
M <- matrix(unlist(u), length(u), 2, byrow = TRUE)
out <- cover.design(M, num.sp.knots)
knots.sp <- out$design

knots.t <- c(2005, 2005.5, 2006, 2006.5, 2007, 2007.5, 2008, 2008.5,
    2009, 2009.5, 2010, 2010.5, 2011, 2011.5, 2012, 2012.5, 2013, 2013.5,
    2014, 2014.5, 2015)
knots <- merge(knots.sp, knots.t)
names(knots) <- c("x", "y", "t")
basis <- SpaceTimeBisquareBasis$new(knots[,1], knots[,2], knots[,3], w.s = radius, w.t = 1)

pdf("design.pdf")
plot(dom.fine[,1], col = "white")
points(knots.sp)
dev.off()

plan.db <- read_csv("../planning_2016//pdb2016bgv8_us.csv")
head(plan.db)
# plan.db <- read.csv("planning/pdb2016bgv8_us.csv")
# head(plan.db)

cb_2016_42_bg <- st_read("cb_2016_42_bg_500k/cb_2016_42_bg_500k.shp")
cb_2016_42_bg <- st_transform(cb_2016_42_bg, crs = st_crs(dom.fine))

plan.db <- plan.db %>%
    mutate(State = as.character(State)) %>%
    mutate(County = as.character(County)) %>%
    mutate(Tract = as.character(Tract)) %>%
    mutate(Block_group = as.character(Block_group)) %>%
    mutate(Med_HHD = sub("\\$", "", Med_HHD_Inc_BG_ACS_10_14)) %>%
    mutate(Med_HHD = sub(",", "", Med_HHD)) %>%
    mutate(Med_HHD = as.numeric(Med_HHD)) %>%
    mutate(Med_HHD_MOE = sub("\\$", "", Med_HHD_Inc_BG_ACSMOE_10_14)) %>%
    mutate(Med_HHD_MOE = sub(",", "", Med_HHD_MOE)) %>%
    mutate(Med_HHD_MOE = as.numeric(Med_HHD_MOE)) %>%
    dplyr::select(State, County, Tract, Block_group, Med_HHD, Med_HHD_MOE, Tot_Housing_Units_CEN_2010)

cb_2016_42_bg <- cb_2016_42_bg %>%
    mutate(STATEFP = as.character(STATEFP)) %>%
    mutate(COUNTYFP = as.character(COUNTYFP)) %>%
    mutate(TRACTCE = as.character(TRACTCE)) %>%
    mutate(BLKGRPCE = as.character(BLKGRPCE)) %>%
    left_join(plan.db, by = c("STATEFP" = "State", "COUNTYFP" = "County", "TRACTCE" = "Tract", "BLKGRPCE" = "Block_group"))

# It takes quite a while to draw from areal units, so we'll just draw the
# locations once at the beginning of the simulation. We'll redraw the survey'd
# variables within each rep

N.total <- sum(cb_2016_42_bg$Tot_Housing_Units_CEN_2010[
    cb_2016_42_bg$Tot_Housing_Units_CEN_2010 > 0 &
    !is.na(cb_2016_42_bg$Med_HHD) &
    !is.na(cb_2016_42_bg$Med_HHD_MOE)])
na <- rep(NA, N.total)
sim.popn <- data.frame(xcoord = na, ycoord = na, state = NA,
    county = NA, tract = NA, block_group = NA, geo_id = NA,
    mean_y = NA, sd_y = NA, y = NA)
geo2units <- list()
last.idx <- 0
for (i in 1:nrow(cb_2016_42_bg)) {
    if (i %% 100 == 0) {
        logger("Generating data for block group %d of %d\n", i, nrow(cb_2016_42_bg))
    }

    N.blg <- cb_2016_42_bg$Tot_Housing_Units_CEN_2010[i]
    if (N.blg == 0 || is.na(cb_2016_42_bg$Med_HHD[i]) || is.na(cb_2016_42_bg$Med_HHD_MOE[i])) {
        logger("Skipping block group %d of %d with GEOID %s\n", i, nrow(cb_2016_42_bg), cb_2016_42_bg$GEOID[i])
    } else {
        pp <- rArea(N.blg, cb_2016_42_bg[i,], blocksize = 2*N.blg)

        idx <- 1:N.blg + last.idx
        sim.popn$xcoord[idx] <- pp[,1]
        sim.popn$ycoord[idx] <- pp[,2]
        sim.popn$state[idx] <- cb_2016_42_bg$STATEFP[i]
        sim.popn$county[idx] <- cb_2016_42_bg$COUNTYFP[i]
        sim.popn$tract[idx] <- cb_2016_42_bg$TRACTCE[i]
        sim.popn$block_group[idx] <- cb_2016_42_bg$BLKGRPCE[i]
        sim.popn$geo_id[idx] <- cb_2016_42_bg$GEOID[i]
        
        sim.popn$mean_y[idx] <- cb_2016_42_bg$Med_HHD[i]
        sim.popn$sd_y[idx] <- cb_2016_42_bg$Med_HHD_MOE[i] / 1.645

        geo2units[[cb_2016_42_bg$GEOID[i]]] <- idx
        last.idx <- last.idx + N.blg
    }
}
save.image("popn.Rdata")

# ----- Construct a STCOSPrep object and add space-time domains -----
# Do this without having generated DirectEst and DirectVar first
year.levels <- 2005:2015
sp <- STCOSPrep$new(fine_domain = dom.fine, fine_domain_geo_name = "GEO_ID", basis = basis, basis_mc_reps = 500)

source.supps.1yr <- list()
for (j in 1:length(year.levels)) {
    year <- year.levels[j]
    supp <- dom.fine
    supp$DirectEst <- NA
    supp$DirectVar <- NA
    lookback <- year
    sp$add_obs(supp, period = lookback, estimate_name = "DirectEst",
        variance_name = "DirectVar", geo_name = "GEO_ID")
    source.supps.1yr[[as.character(year)]] <- supp
}

source.supps.3yr <- list()
for (j in 1:length(year.levels)) {
    if (j < 3) next
    year <- year.levels[j]
    supp <- dom.fine
    supp$DirectEst <- NA
    supp$DirectVar <- NA
    lookback <- seq(year-2, year)
    sp$add_obs(supp, period = lookback, estimate_name = "DirectEst",
        variance_name = "DirectVar", geo_name = "GEO_ID")
    source.supps.3yr[[as.character(year)]] <- supp
}

source.supps.5yr <- list()
for (j in 1:length(year.levels)) {
    if (j < 5) next
    year <- year.levels[j]
    supp <- dom.fine
    supp$DirectEst <- NA
    supp$DirectVar <- NA
    lookback <- seq(year-4, year)
    sp$add_obs(supp, period = lookback, estimate_name = "DirectEst",
        variance_name = "DirectVar", geo_name = "GEO_ID")
    source.supps.5yr[[as.character(year)]] <- supp
}

# ----- Dimension reduction -----
logger("Runing dimension reduction on S")
S <- sp$get_S()
eig <- eigen(t(S) %*% S)
rho <- eig$values

pdf("cols.pdf", width = 5, height = 5)
plot(cumsum(rho) / sum(rho), ylab = "Proportion of Eigenvalues", xlab = "Number of Eigenvalues")

idx.S <- which(cumsum(rho) / sum(rho) < 0.60)
abline(v = max(idx.S), lty = 2, col = "red", lwd = 2)
axis(side = 3, at = max(idx.S), lwd = 2, padj = 0.5, hadj = 0.75)

idx.S <- which(cumsum(rho) / sum(rho) < 0.75)
abline(v = max(idx.S), lty = 2, col = "red", lwd = 2)
axis(side = 3, at = max(idx.S), lwd = 2, padj = -1)

idx.S <- which(cumsum(rho) / sum(rho) < 0.90)
abline(v = max(idx.S), lty = 2, col = "red", lwd = 2)
axis(side = 3, at = max(idx.S), lwd = 2, padj = 0.5)
dev.off()

idx.S <- which(cumsum(rho) / sum(rho) < pr.rho)
Tx.S <- t(eig$vectors[idx.S,])
f <- function(S) { S %*% Tx.S }
sp$set_basis_reduction(f)

# ----- Compute Kinv -----
S.reduced <- sp$get_reduced_S()
K.inv <- diag(x = 1, nrow = ncol(S.reduced))

# ----- Set up target supports -----
logger("Setting up target geography\n")
cb_2016_us_cd115 <- st_read("cb_2016_us_cd115_500k/cb_2016_us_cd115_500k.shp") %>%
    filter(STATEFP == '42') %>%
    st_transform(crs = st_crs(dom.fine))
# cb_2016_us_cd115 <- st_transform(cb_2016_us_cd115, crs = st_crs(dom.fine))
plot(dom.fine[,1])
plot(cb_2016_us_cd115[,1], col = "NA", add = TRUE, lwd = 2, lty = 2)
plot(cb_2016_us_cd115[,4])

# See https://gis.stackexchange.com/questions/225157/generate-rectangular-fishnet-or-vector-grid-cells-shapefile-in-r/243585
# Remove squares that intersect with the outside of the state
# Also remove Erie County (049) because much of it is water
state.grid <- st_make_grid(extract(dom.fine), n = c(gridn, gridn), what = 'polygons') %>%
    st_sf('geometry' = ., data.frame('GEO_ID' = 1:length(.))) %>%
    st_transform(st_crs(dom.fine))
d1 <- state.grid %>%
    st_join(county_acs_5yr2015) %>%
    filter(STATE != '42' | COUNTY == '049' | is.na(COUNTY)) %>%
    group_by(GEO_ID.x) %>%
    summarize(count = length(GEO_ID.x)) %>%
    mutate(GEO_ID = GEO_ID.x) %>%
    select(GEO_ID, count)
idx.drop <- which(state.grid$GEO_ID %in% d1$GEO_ID)
state.grid <- state.grid[-idx.drop,]
plot(dom.fine[,1])
plot(state.grid[,1], col = NA, add = TRUE)

# See https://gis.stackexchange.com/questions/225157/generate-rectangular-fishnet-or-vector-grid-cells-shapefile-in-r/243585
# Remove squares that intersect with the outside of the state
# Also remove Erie County (049) because much of it is water
state.grid.finer <- st_make_grid(extract(dom.fine), n = c(gridn*2, gridn*2), what = 'polygons') %>%
    st_sf('geometry' = ., data.frame('GEO_ID' = 1:length(.))) %>%
    st_transform(st_crs(dom.fine))
d1 <- state.grid.finer %>%
    st_join(county_acs_5yr2015) %>%
    filter(STATE != '42' | COUNTY == '049' | is.na(COUNTY)) %>%
    group_by(GEO_ID.x) %>%
    summarize(count = length(GEO_ID.x)) %>%
    mutate(GEO_ID = GEO_ID.x) %>%
    select(GEO_ID, count)
idx.drop <- which(state.grid.finer$GEO_ID %in% d1$GEO_ID)
state.grid.finer <- state.grid.finer[-idx.drop,]
plot(dom.fine[,1])
plot(state.grid.finer[,1], col = NA, add = TRUE)

# See https://gis.stackexchange.com/questions/225157/generate-rectangular-fishnet-or-vector-grid-cells-shapefile-in-r/243585
# Remove squares that intersect with the outside of the state
# Also remove Erie County (049) because much of it is water
state.grid.coarser <- st_make_grid(extract(dom.fine), n = c(gridn/2, gridn/2), what = 'polygons') %>%
    st_sf('geometry' = ., data.frame('GEO_ID' = 1:length(.))) %>%
    st_transform(st_crs(dom.fine))
d1 <- state.grid.coarser %>%
    st_join(county_acs_5yr2015) %>%
    filter(STATE != '42' | COUNTY == '049' | is.na(COUNTY)) %>%
    group_by(GEO_ID.x) %>%
    summarize(count = length(GEO_ID.x)) %>%
    mutate(GEO_ID = GEO_ID.x) %>%
    select(GEO_ID, count)
idx.drop <- which(state.grid.coarser$GEO_ID %in% d1$GEO_ID)
state.grid.coarser <- state.grid.coarser[-idx.drop,]
plot(dom.fine[,1])
plot(state.grid.coarser[,1], col = NA, add = TRUE)

target.out <- sp$domain2model(cb_2016_us_cd115, period = 2016, geo_name = 'GEOID')
H.targ <- target.out$H
S.reduced.targ <- target.out$S.reduced

target2.out <- sp$domain2model(dom.fine, period = 2016, geo_name = 'GEO_ID')
H.targ2 <- target2.out$H
S.reduced.targ2 <- target2.out$S.reduced

target3.out <- sp$domain2model(state.grid, period = 2016, geo_name = 'GEO_ID')
H.targ3 <- target3.out$H
S.reduced.targ3 <- target3.out$S.reduced

target4.out <- sp$domain2model(state.grid.finer, period = 2016, geo_name = 'GEO_ID')
H.targ4 <- target4.out$H
S.reduced.targ4 <- target4.out$S.reduced

target5.out <- sp$domain2model(state.grid.coarser, period = 2016, geo_name = 'GEO_ID')
H.targ5 <- target5.out$H
S.reduced.targ5 <- target5.out$S.reduced

RES1 <- matrix(NA, nrow(H.targ), reps)
RES2 <- matrix(NA, nrow(H.targ2), reps)
RES3 <- matrix(NA, nrow(H.targ3), reps)
RES4 <- matrix(NA, nrow(H.targ4), reps)
RES5 <- matrix(NA, nrow(H.targ5), reps)

for (rep in 1:reps)
{
    logger("Starting simulation rep %d\n", rep)

    logger("Generating unit level data\n")
    # We'll generate data at the block-group level, but model at the county level.
    # Need population sizes for each block-group
    # Need average and stddev income for block-groups
    # Need shape file for block-groups

    # Assume mean_y and sd_y are parameters for lognormal rvs.
    # Solve for mean and sd of normal rvs.
    # Then generate normal rvs representing the data on a log scale
    var.normal <- log(1 + sim.popn$sd_y^2 / sim.popn$mean_y^2)
    mean.normal <- log(sim.popn$mean_y) - 1/2 * var.normal
    sim.popn$y <- rnorm(nrow(sim.popn), mean.normal, sqrt(var.normal))
    # hist(exp(sim.popn$y[sim.popn$y < log(200000)]))

    # Aggregate simulated y to each target support
    totals1 <- sim.popn %>%
        st_as_sf(coords = c("xcoord", "ycoord"), crs = st_crs(dom.fine)) %>%
        st_join(cb_2016_us_cd115) %>%
        group_by(state, CD115FP) %>%
        summarize(y_bar = mean(y), N = length(y))
    st_geometry(totals1) <- NULL
    order(cb_2016_us_cd115$CD115FP)
    totals1 <- totals1[order(cb_2016_us_cd115$CD115FP),]

    totals2 <- sim.popn %>%
        group_by(state, county) %>%
        summarize(y_bar = mean(y), N = length(y))
    totals2 <- totals2[order(dom.fine$COUNTY),]

    totals3 <- sim.popn %>%
        st_as_sf(coords = c("xcoord", "ycoord"), crs = st_crs(dom.fine)) %>%
        st_join(state.grid) %>%
        filter(!is.na(GEO_ID)) %>%
        group_by(GEO_ID) %>%
        summarize(y_bar = mean(y), N = length(y))
    st_geometry(totals3) <- NULL
    totals3 <- totals3[order(state.grid$GEO_ID),]

    totals4 <- sim.popn %>%
        st_as_sf(coords = c("xcoord", "ycoord"), crs = st_crs(dom.fine)) %>%
        st_join(state.grid.finer) %>%
        filter(!is.na(GEO_ID)) %>%
        group_by(GEO_ID) %>%
        summarize(y_bar = mean(y), N = length(y))
    st_geometry(totals4) <- NULL
    totals4 <- totals4[order(state.grid.finer$GEO_ID),]

    totals5 <- sim.popn %>%
        st_as_sf(coords = c("xcoord", "ycoord"), crs = st_crs(dom.fine)) %>%
        st_join(state.grid.coarser) %>%
        filter(!is.na(GEO_ID)) %>%
        group_by(GEO_ID) %>%
        summarize(y_bar = mean(y), N = length(y))
    st_geometry(totals5) <- NULL
    totals5 <- totals5[order(state.grid.coarser$GEO_ID),]


    # ----- Take a sample to get estimates and variances -----
    # First let's assume that all estimates can be disclosed.
    logger("Sample to get estimates\n")

    # TBD: Use appropriate HT estimates for the mean and variance
    logger("Taking samples for 1-year estimates\n")
    sizes <- sim.popn %>%
        group_by(county) %>%
        summarize(popn = length(y)) %>%
        mutate(sample = ceiling(sample.prop * popn))
    na <- rep(NA, sum(sizes$sample) * length(year.levels))
    all.samples <- data.frame(year = na, county = na, unit = na) %>%
        mutate(county = as.character(county))
    last.idx <- 0
    for (j in 1:length(year.levels)) {
        year <- year.levels[j]
        for (l in 1:nrow(supp)) {
            county <- supp$COUNTY[l]
            idx.county <- which(sim.popn$county == county)
            idx.sample <- sample(idx.county, size = sizes$sample[l], replace = TRUE)
            source.supps.1yr[[as.character(year)]]$DirectEst[l] <- mean(sim.popn$y[idx.sample])
            source.supps.1yr[[as.character(year)]]$DirectVar[l] <- var(sim.popn$y[idx.sample])

            ind <- 1:sizes$sample[l] + last.idx
            all.samples$year[ind] <- year
            all.samples$county[ind] <- county
            all.samples$unit[ind] <- idx.sample
            last.idx <- last.idx + sizes$sample[l]
        }
    }

    logger("Taking samples for 3-year estimates\n")
    for (j in 1:length(year.levels)) {
        if (j < 3) { next }
        year <- year.levels[j]
        lookback <- year.levels[seq(j-2, j)]
        sample.3yr <- all.samples %>%
            filter(year %in% lookback)
        dat.estimates <- sim.popn[sample.3yr$unit,] %>%
            group_by(county) %>%
            summarize(direct_est = mean(y), direct_var = var(y))
        supp <- dom.fine %>%
            mutate(COUNTY = as.character(COUNTY)) %>%
            inner_join(dat.estimates, by = c("COUNTY" = "county")) %>%
            mutate(DirectEst = direct_est) %>%
            mutate(DirectVar = direct_var) %>%
            dplyr::select(-c(direct_est, direct_var))
        source.supps.3yr[[as.character(year)]]$DirectEst <- supp$DirectEst
        source.supps.3yr[[as.character(year)]]$DirectVar <- supp$DirectVar
    }

    logger("Taking samples for 5-year estimates\n")
    for (j in 1:length(year.levels)) {
        if (j < 5) { next }
        year <- year.levels[j]
        lookback <- year.levels[seq(j-4, j)]
        sample.5yr <- all.samples %>%
            filter(year %in% lookback)
        dat.estimates <- sim.popn[sample.5yr$unit,] %>%
            group_by(county) %>%
            summarize(direct_est = mean(y), direct_var = var(y))
        supp <- dom.fine %>%
            mutate(COUNTY = as.character(COUNTY)) %>%
            inner_join(dat.estimates, by = c("COUNTY" = "county")) %>%
            mutate(DirectEst = direct_est) %>%
            mutate(DirectVar = direct_var) %>%
            dplyr::select(-c(direct_est, direct_var))
        source.supps.5yr[[as.character(year)]]$DirectEst <- supp$DirectEst
        source.supps.5yr[[as.character(year)]]$DirectVar <- supp$DirectVar
    }

    # save.image("temp.Rdata")

    Z <- c(
        unlist(Map(f = function(x) { x$DirectEst }, source.supps.1yr)),
        unlist(Map(f = function(x) { x$DirectEst }, source.supps.3yr)),
        unlist(Map(f = function(x) { x$DirectEst }, source.supps.5yr))
    )
    V <- c(
        unlist(Map(f = function(x) { x$DirectVar }, source.supps.1yr)),
        unlist(Map(f = function(x) { x$DirectVar }, source.supps.3yr)),
        unlist(Map(f = function(x) { x$DirectVar }, source.supps.5yr))
    )
    H <- sp$get_H()

    # Stdize before MCMC
    D <- Diagonal(n = length(Z), x = 1/sd(Z))
    Z.scaled <- (Z - mean(Z)) / sd(Z)
    V.scaled <- V / var(Z)

    # ----- Run MCMC -----
    # Compute MLE as initial value
    mle.out <- mle.stcos(Z.scaled, S.reduced, V.scaled, H, init = list(sig2xi = 1))
    init <- list(
        sig2xi = mle.out$sig2xi.hat,
        mu_B = mle.out$mu.hat,
        eta = mle.out$eta.hat
    )
    gibbs.out <- gibbs.stcos.raw(Z.scaled, S.reduced, V.scaled, K.inv, H,
        R = 10000, report.period = 1000, burn = 1000, thin = 10,
        init = init)
    if (FALSE) {
        print(gibbs.out)
        plot(sig2mu.mcmc <- mcmc(gibbs.out$sig2mu.hist))
        plot(sig2xi.mcmc <- mcmc(gibbs.out$sig2xi.hist))
        plot(sig2K.mcmc <- mcmc(gibbs.out$sig2K.hist))
    }

    # Save results for this rep. At the end we can compare posterior expectation
    # of Y on each target support of interest with point level data. For example,
    # we can compute the true mean at the target support level, or recompute
    # estimates of the mean using sampled data.
    alpha <- 0.05

    # Posterior distribution for E(Y), applied to target support 1 (CDs)
    E.hat.scaled <- fitted(gibbs.out, H.targ, S.reduced.targ)
    E.hat <- sd(Z) * E.hat.scaled + mean(Z)
    RES1[,rep] <- colMeans(E.hat)
    if (FALSE) {
        cb_2016_us_cd115$E.mean <- colMeans(E.hat)
        cb_2016_us_cd115$E.sd <- apply(E.hat, 2, sd)
        cb_2016_us_cd115$E.lo <- apply(E.hat, 2, quantile, prob = alpha/2)
        cb_2016_us_cd115$E.hi <- apply(E.hat, 2, quantile, prob = 1 - alpha/2)
        plot(exp(cb_2016_us_cd115$y_bar), exp(cb_2016_us_cd115$E.mean))
        plot(cb_2016_us_cd115$y_bar, cb_2016_us_cd115$E.mean)
        plot(cb_2016_us_cd115[,c("y_bar","E.mean")])
    }
    
    # Posterior distribution for E(Y), applied to target support 2 (counties)
    E.hat.scaled <- fitted(gibbs.out, H.targ2, S.reduced.targ2)
    E.hat <- sd(Z) * E.hat.scaled + mean(Z)
    RES2[,rep] <- colMeans(E.hat)
    if (FALSE) {
        dom.fine$E.mean <- colMeans(E.hat)
        dom.fine$E.sd <- apply(E.hat, 2, sd)
        dom.fine$E.lo <- apply(E.hat, 2, quantile, prob = alpha/2)
        dom.fine$E.hi <- apply(E.hat, 2, quantile, prob = 1 - alpha/2)
        plot(exp(dom.fine$y_bar), exp(dom.fine$E.mean))
        plot(dom.fine$y_bar, dom.fine$E.mean)
        plot(dom.fine[,c("y_bar","E.mean")])
    }

    # Posterior distribution for E(Y), applied to target support 3 (grid spaces)
    E.hat.scaled <- fitted(gibbs.out, H.targ3, S.reduced.targ3)
    E.hat <- sd(Z) * E.hat.scaled + mean(Z)
    RES3[,rep] <- colMeans(E.hat)
    if (FALSE) {
        state.grid$E.mean <- colMeans(E.hat)
        state.grid$E.sd <- apply(E.hat, 2, sd)
        state.grid$E.lo <- apply(E.hat, 2, quantile, prob = alpha/2)
        state.grid$E.hi <- apply(E.hat, 2, quantile, prob = 1 - alpha/2)
        plot(exp(state.grid$y_bar), exp(state.grid$E.mean))
        plot(state.grid$y_bar, state.grid$E.mean)
        plot(state.grid[,c("y_bar","E.mean")])
    }

    # Posterior distribution for E(Y), applied to target support 4 (finer grid spaces)
    E.hat.scaled <- fitted(gibbs.out, H.targ4, S.reduced.targ4)
    E.hat <- sd(Z) * E.hat.scaled + mean(Z)
    RES4[,rep] <- colMeans(E.hat)
    if (FALSE) {
        state.grid.finer$E.mean <- colMeans(E.hat)
        state.grid.finer$E.sd <- apply(E.hat, 2, sd)
        state.grid.finer$E.lo <- apply(E.hat, 2, quantile, prob = alpha/2)
        state.grid.finer$E.hi <- apply(E.hat, 2, quantile, prob = 1 - alpha/2)
        plot(exp(state.grid.finer$y_bar), exp(state.grid.finer$E.mean))
        plot(state.grid.finer$y_bar, state.grid.finer$E.mean)
        plot(state.grid.finer[,c("y_bar","E.mean")])
    }

    # Posterior distribution for E(Y), applied to target support 5 (coarser grid spaces)
    E.hat.scaled <- fitted(gibbs.out, H.targ5, S.reduced.targ5)
    E.hat <- sd(Z) * E.hat.scaled + mean(Z)
    RES5[,rep] <- colMeans(E.hat)
    if (FALSE) {
        state.grid.coarser$E.mean <- colMeans(E.hat)
        state.grid.coarser$E.sd <- apply(E.hat, 2, sd)
        state.grid.coarser$E.lo <- apply(E.hat, 2, quantile, prob = alpha/2)
        state.grid.coarser$E.hi <- apply(E.hat, 2, quantile, prob = 1 - alpha/2)
        plot(exp(state.grid.coarser$y_bar), exp(state.grid.coarser$E.mean))
        plot(state.grid.coarser$y_bar, state.grid.coarser$E.mean)
        plot(state.grid.coarser[,c("y_bar","E.mean")])
    }

    # save.image("temp.Rdata")
}

totals1$E_hat <- RES1
totals2$E_hat <- RES2
totals3$E_hat <- RES3
totals4$E_hat <- RES4
totals5$E_hat <- RES5
totals1$err <- RES1 - totals1$y_bar
totals2$err <- RES2 - totals2$y_bar
totals3$err <- RES3 - totals3$y_bar
totals4$err <- RES4 - totals4$y_bar
totals5$err <- RES5 - totals5$y_bar

totals1_sf <- cb_2016_us_cd115 %>%
    inner_join(totals1, by = c("STATEFP" = "state", "CD115FP" = "CD115FP"))
totals2_sf <- dom.fine %>%
    inner_join(totals2, by = c("STATE" = "state", "COUNTY" = "county"))
totals3_sf <- state.grid %>%
    inner_join(totals3, by = c("GEO_ID" = "GEO_ID"))
totals4_sf <- state.grid.finer %>%
    inner_join(totals4, by = c("GEO_ID" = "GEO_ID"))
totals5_sf <- state.grid.coarser %>%
    inner_join(totals5, by = c("GEO_ID" = "GEO_ID"))

plot(totals1_sf[,c("y_bar", "E_hat")])
plot(totals2_sf[,c("y_bar", "E_hat")])
plot(totals3_sf[,c("y_bar", "E_hat")])
plot(totals4_sf[,c("y_bar", "E_hat")])
plot(totals5_sf[,c("y_bar", "E_hat")])

plot(totals1_sf[,c("err")])
plot(totals2_sf[,c("err")])
plot(totals3_sf[,c("err")])
plot(totals4_sf[,c("err")])
plot(totals5_sf[,c("err")])

plot(totals1_sf$y_bar, totals1_sf$E_hat); abline(c(0,1))
plot(totals2_sf$y_bar, totals2_sf$E_hat); abline(c(0,1))
plot(totals3_sf$y_bar, totals3_sf$E_hat); abline(c(0,1))
plot(totals4_sf$y_bar, totals4_sf$E_hat); abline(c(0,1))
plot(totals5_sf$y_bar, totals5_sf$E_hat); abline(c(0,1))

save.image("results.Rdata")
\end{verbatim}
\end{footnotesize}


\bibliographystyle{plainnat}
\bibliography{references}
%\input{references.bbl}

\end{document}
